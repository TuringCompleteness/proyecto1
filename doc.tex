\documentclass[11pt]{report}
\usepackage[spanish]{babel}
\usepackage[utf8]{inputenc}
\usepackage{amsmath}
\usepackage{amssymb}

\begin{document}{}
\title{Proyecto intermedio}
\author{
  Arteaga Vázquez Alan Ernesto \
  \and
  Coronel Ruiz Aldair \
  \and
  García Pérez Adrián \
}\date{}
\maketitle

\textbf{Introducción} \\

Se busca desarrollar una aplicación móvil en el sistema operativo
Android, utilizando los conceptos vistos en clase sobre dicho entorno
de desarrollo.

La aplicación mostrará una lista de las agrupaciones de super
héroes más representativas del universo Marvel en la actualidad.
Dentro de la aplicación, será posible navegar por cada agrupación para obtener
una lista de sus miembros, así como sus información adicional sobre los mismos. \\ \\ 

\textbf{Público} \\

La aplicación está destinada para todas las personas que estén interesadas
en conocer con más detalle a las agrupaciones de super héroes más importantes
en la actualidad. El sector poblacional comprende niños, adolescentes y adultos
con o sin conocimiento previo de los distintos super héroes del universo Marvel. \\

\textbf{Alcance y desarrollo} \\

Una primera versión de la aplicación consistió en mostrar diversas agrupaciones de
super héroes y algunos miembros, sin consumir datos de la API de Marvel. \\

Posteriormente, se hicieron las primeras aproximaciones para comunicar la aplicación 
con la API, logrando obtener información básica pero sin mostrarla propiamente en 
las vistas. \\

Como última versión, se implementaron las vistas para presentar la información de
la API al usuario, como imágenes de los super héroes y una descripción de los
mismos, con una pantalla dedicada. \\

A lo largo del desarrollo, se utilizaron distintas tecnologías que resultaron
útiles para facilitar el trabajo. Se utilizó Retrofit, para la comunicación
con la API de Marvel; se utilizó Gson, para poder utilizar de manera sencilla
los objetos JSON recibidos con Retrofit; por último se utilizó Glide, para
mostrar imágenes de los super héroes consumiéndolas el respectivo recurso 
en internet. \\

Como planes a futuro, buscamos mejorar la aplicación para brindarle una
mejor experiencia al usuario, mostrando más información sobre los héroes
y contar con un mayor número de ellos, a pesar de que una posible 
limitación sería la cantidad de héroes disponibles en la API de Marvel.


\end{document}
